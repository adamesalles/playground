\documentclass[12pt]{article}
\usepackage[utf8]{inputenc}
\usepackage[brazil]{babel}
\usepackage{newpxtext}
\usepackage[a4paper, margin=1.5cm]{geometry}
\usepackage[shortlabels]{enumitem}
\usepackage{indentfirst}

\title{Carta para recorrência de multa}
\author{Ernaldo Dantas Bezerra}

\begin{document}

% \maketitle
\thispagestyle{empty}
 
No dia 12 de abril de 2024, foi realizada uma inspeção por iniciativa da CPFL Piratininga no medidor de consumo de energia elétrica na minha residência, com o número de instalação 4001695503. Esta inspeção, conforme descrita no Termo de Ocorrência e Inspeção (TOI) de número 799722087, resultou em uma multa por suposto desvio de energia elétrica, que não condiz com a realidade.

A inspeção foi realizada sem a minha autorização e com uma série de irregularidades. O TOI e todos os documentos que relatam a inspeção estão assinados por Luisa Maria Gonçalves, que consta como consumidora responsável pela residência. Entretanto, essa pessoa é desconhecida por mim e tampouco tem autorização para com a minha propriedade, o que invalida qualquer permissão ou solicitação conferida.

Além disso, eu me fiz presente no momento da inspeção e não fui procurado para assinar qualquer documento ou para acompanhar o procedimento. No TOI, consta que a caixa de medição foi encontrada sem lacre, o que não procede. O lacre foi rompido durante a inspeção e não estava danificado antes da visita. Ainda, um cabo azul foi instalado de forma irregular pelos inspetores, sem qualquer autorização ou conhecimento prévio, e após a vistoria foi removido. Esse cabo azul foi apontado como o suposto desvio de energia, como consta nos registros fotográficos anexados ao TOI. 

Apenas o meu filho mora nessa residência, na qual ele se ausenta entre às 7h e 19h para trabalhar. A energia é utilizada apenas para o consumo de eletrodomésticos básicos, como geladeira, televisão, computador e lâmpadas. Não há qualquer atividade que justifique um desvio de energia elétrica, muito menos um gasto excessivo. Além do mais, a instalação é de baixa renda.

Mediante a situação relatada, solicito a revisão da multa aplicada, pois a mesma não procede. Estou à disposição para esclarecer qualquer dúvida e para colaborar com as investigações necessárias.

PS: Segue em anexo contrato de compra e venda, que comprova o nome do proprietário do terreno, que difere da sra. Luisa Maria Gonçalves; e a conta de água, que justifica o baixo consumo residencial.

\vspace*{5mm}
Atenciosamente,

Ernaldo Dantas Bezerra.

\begin{table}[!h]
    \centering
    \vspace{15mm}
    \begin{tabular}{@{}p{3in}p{2in}p{2in}@{}}
        \hrulefill &&\\
        \hspace*{1.2cm} Assinatura do consumidor &&\\
    \end{tabular}
\end{table}
\end{document}